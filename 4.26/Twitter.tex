\documentclass[fleqn, 14pt]{extarticlej}
\oddsidemargin=-1cm
\usepackage[dvipdfm]{graphicx}
\textwidth=18cm
\textheight=23cm
\topmargin=0cm
\headheight=1cm
\headsep=0cm
\footskip=1cm

\def\labelenumi{(\theenumi)}
\def\theenumii{\Alph{enumii}}
\def\theenumiii{(\alph{enumiii})}
\usepackage{comment}
\usepackage{indentfirst}
\begin{document}
%%%%%%%%%%%%%%%%%%%%%%%%%%%%
%% 表題
%%%%%%%%%%%%%%%%%%%%%%%%%%%%
\begin{center}
{\Large {\bf TwitterBotプログラムの仕様書}}

\end{center}
\begin{flushright}
2013年 4月 26日

乃村研究室 河野 達生
\end{flushright}

\section{概要 }
本資料は,平成25年度GNグループの新人研修課題で作成したTwitteBotプログラムの仕様書である.


\section{機能}
\subsection{基本機能}
本プログラムは,以下の2つの機能をもつ.
\begin{description}
  \item[機能1] 任意の文字列をツイートする機能
  \item[機能2] ツイートを受信する機能
\end{description}

\subsection{追加機能}
追加機能として,以下の2つの機能を追加した.
\begin{description}
  \item[機能3] 受信したツイートの中に“「○○○」と言って”という文字列があった場合は,“○○○”とツイートする機能
  \item[機能4] 乃村研究室のメンバの誕生日をツイートする機能
\end{description}


\section{動作環境}
本プログラムの動作環境を表1に示す.
\begin{table}[h]
  \begin{center}
    \caption{動作環境}
    \begin{tabular}{c|c}\hline\hline
      名前  & version\\ \hline
      Ruby   &  1.9.3 \\ 
      OAuth &  0.4.7 \\ \hline
    \end{tabular} 
  \end{center}
\end{table}

\section{動作確認済み環境}
本プログラムの動作を確認した環境を表2に示す.
\begin{table}[h]
  \begin{center}
    \caption{動作確認済み環境}
    \begin{tabular}{c|c}\hline\hline
      項目  & 詳細 \\ \hline
       OS     & Windows7 Home Premium 64bit \\ 
      CPU   & Intel(R) Core(TM) i3-2100 CPU @ 3.10GHz\\ 
      メモリ   & 4096MB \\ 
      ブラウザ & Firefox 20.0.1 \\ \hline
    \end{tabular} 
  \end{center}
\end{table}
  
\section{使用方法}
本プログラムを仮想計算機kawasemiで実行する方法を以下に示す.
\begin{enumerate}
  \item 仮想計算機kawasemiに接続する.
  \item \verb|MyTwitterBot.rb|が存在するディレクトリに移動する.
  \item \verb|MyTwitterBot.rb|内の13行目の括弧内にツイートしたい文字列を入力する.
  \item 同じディレクトリ内にbirth.ymlというファイルを作成し,以下の形式で誕生日の情報を記入する.\\
   \hspace{1.0zw}- Name : ユーザ名\\
   \hspace{1.5zw}  Id   : ユーザID\\
   \hspace{1.5zw}  Year : 誕生年\\
   \hspace{1.5zw}  Month: 誕生月\\
   \hspace{1.5zw}  Day  : 誕生日\\
   

  \item 以下のコマンドによってプログラムを実行する.\\
  \hspace{1.0zw}\% ruby MyTwitterBot.rb\\
\end{enumerate}

\section{エラー処理と保障しない動作}
\subsection{ エラー処理}
想定されるエラー2つと各エラーに対する処理を以下に記述する.
\begin{enumerate}
  \item 機能1の文字数が140文字以上のメッセージの場合\\
    (処理)Twitterの仕様上,140文字以上の文字列をツイートすることができない[1].このため,140文字以上の文字列があった場合,ツイートされない.

  \item birth.yml が存在しない場合\\
    (処理)birth.ymlが存在するかしないかの条件処理を行う.存在する場合は,“birth.ymlが存在します.”とメッセージを出力し,実行する.存在しない場合は,“birth.ymlが存在しません.”とメッセージを出力し,プログラムを終了させる.
\end{enumerate}

\subsection{保証しない動作}
保証しない動作2つを以下に記述する.
\begin{enumerate}

 \item 短時間に同じ内容のツイートを行う.\\
   Twitterの仕様上,短時間に同じメッセージをツイートすることができない.

 \item 1日に1000件以上のツイートを行う.\\
   Twitterの仕様上,1日にツイートできる回数は1000件までである.

\end{enumerate}
\begin{thebibliography}{99}
  \bibitem {book1} Twitter:Twitterとは,入手先$<$https://business.twitter.com/ja/basics/what-is-twitter$>$(参照 2013-4-25)
\end{thebibliography}


\end{document}
